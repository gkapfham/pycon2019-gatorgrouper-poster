%%%% Better Poster latex template example v1.0 (2019/04/04)
%%%% GNU General Public License v3.0
%%%% Rafael Bailo
%%%% https://github.com/rafaelbailo/betterposter-latex-template
%%%%
%%%% Original design from Mike Morrison
%%%% https://twitter.com/mikemorrison

\documentclass[a0paper,fleqn]{betterposter}
\usepackage{float}
\usepackage[latin1]{inputenc}
\usepackage{tikz}
\usepackage{booktabs}
\usetikzlibrary{shapes,arrows}
\usetikzlibrary{arrows.meta}

\usepackage{enumitem}
\usepackage{fontawesome}

\newcommand{\codedatagit}{\faCode~+~\faDatabase~in \faGit}

\tikzstyle{block2} = [rectangle, draw, fill=white!20,
text width=6em, text centered, rounded corners, minimum height=4em, node distance = 10em, text=black]

\begin{document}

\betterposter{

  %%%%%%%% MAIN COLUMN

  \maincolumn{

    %% Main space

    The \textbf{GatorGrouper} tool supports effective teamwork by automatically
    forming teams
    %
    \vfill

    \vspace*{-.5in}
    %
    % \include{fiasco}
    %
    \vspace*{-2.5in}

    Does the status of $T_i$ change when Function-Fiasco does not run $m$?
    %
    \vspace*{-4.5in}

    }{

    %%%% Bottom space

    %% QR code
    %
    \qrcode{img/qr_code}{img/smartphoneWhite}{\textbf{Scan the QR Code} to
    \\visit our GitHub project}

  }

  }{

  %%%%%%%% LEFT COLUMN

  \title{\fontsize{50}{60}\selectfont GatorGrouper: \\ \mbox{Using Python} and GitHub for
  Team Form\-ation and Assessment}
    %
    \author{Gregory M.\@ Kapfhammer}

  \vspace*{-.5in}

  \section{Introduction}
  %
  Python programs are often complex and difficult to test.
  %
  But test coverage does not show that a method was adequately tested.
  %
  A method $m$ is psuedo-tested if a test passes even when $m$ is not run.

  \vspace*{.25in}

  Since it may be time-consuming and error-prone to manually detect
  psuedo-tested methods, Function-Fiasco automatically discovers them for
  engineers.

  \vspace*{-.25in}
  %
  \section{Implementation}
  %
  Function-Fiasco uses technologies like:

  \begin{itemize}

    \item Pytest
    \item Coverage
    \item Decorators
    \item Instrumentation

  \end{itemize}

  \vspace*{.5in}
  %
  Function-Fiasco performs these steps:

  \begin{enumerate}[leftmargin=.5in]

    \item Instrument all program methods
    \item Elide execution of chosen method
    \item Run the tests and observe behavior
    \item Run steps (2) and (3) for all methods
    \item Report the psuedo-tested methods

  \end{enumerate}

  \vspace*{.25in}
  %
  Steps are optimized, ensuring that the tool scales to large Python programs.
  %
  \vspace*{.5in}
  %

  %% This fills the space between the content and the logo
  \vfill

  \includegraphics[width=\textwidth]{img/CollegeLogo.eps}\\

  }{

  %%%%%%%% RIGHT COLUMN

  \vspace*{-.5in}
  %
  \section{Preliminary Results}

  \vspace*{-1in}

  % \include{table}

  \vspace*{-.5in}

  Function-Fiasco detects pseudo-tested methods in real Python programs,
  suggesting the need for improved testing.

  \section{Future Work}
  %
  Add new features to Function-Fiasco: \\
  %
  \vspace*{-.5in}

  \begin{itemize}[leftmargin=*]

    \item{Handle more kinds of methods}
      %
    \item{Improve type fuzzing capability}
      %
    \item{Better observe parameterized tests}
      %
    \item{Report more types of test coverage}

  \end{itemize}

  \vspace{.5em}
  %
  Use improved Function-Fiasco to detect and improve pseudo-tested methods.

  \section{Conclusion}
  %
  Pseudo-tested methods exist in many real-world Python programs.
  %
  Function-Fiasco automatically detects these methods, saving time that testers
  can instead devote to improving test suites.
  %
  Available on GitHub, Function-Fiasco aids the implementation of high-quality
  Pytest test suites and Python programs.

  \section{Get Involved}
  %
  If you would like support the development of Function-Fiasco, please raise an
  issue on the tracker or create a pull request to add a new feature or bug
  fix.
  %
  \vfill

  \section{Acknowledgements}
  %
  Poster creation aided by Cory Wiard.\\
  Feedback provided by Aravind Mohan.\\

  % \vspace*{.5in}
  % \includegraphics[width=\textwidth]{img/ComputerScience-Stack}

}

\end{document}
