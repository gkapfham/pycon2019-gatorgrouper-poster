%%%% Better Poster latex template example v1.0 (2019/04/04)
%%%% GNU General Public License v3.0
%%%% Rafael Bailo
%%%% https://github.com/rafaelbailo/betterposter-latex-template
%%%%
%%%% Original design from Mike Morrison
%%%% https://twitter.com/mikemorrison

\documentclass[a0paper,fleqn]{betterposter}
\usepackage{float}
\usepackage[latin1]{inputenc}
\usepackage{tikz}
\usepackage{booktabs}
\usetikzlibrary{shapes,arrows}
\usetikzlibrary{arrows.meta}

\usepackage{enumitem}
\usepackage{fontawesome}

\begin{document}

\betterposter{

  %%%%%%%% MAIN COLUMN

  \maincolumn{

    % Challenge

    {\fontsize{116}{150}\selectfont

      \faBullhorn~Even though teamwork is central to software engineering
      education, forming groups is difficult!

    }

    \vspace*{-1in}
    \par\noindent\rule{.25\textwidth}{8pt}
    \vspace*{1in}

    % Solution

    {\fontsize{116}{150}\selectfont

    \faLightbulbO~Ready for classroom adoption, \textbf{GatorGrouper} uses data
    to automatically form effective teams

    }

    \vspace*{-1in}
    \par\noindent\rule{.25\textwidth}{8pt}
    \vspace*{1.5in}

    % Equation

    {\fontsize{180}{80}\selectfont

      \faUser{} \, + \,
      \faDatabase{} \, + \,
      \faCode{} \, = \,
      \faUsers{}

    \vspace*{-5in}

    }

    }{

    %%%% Bottom space

    %% QR code
    %
    \qrcode{img/qr_code}{img/smartphoneWhite}{\textbf{Scan the QR Code} to
    \\visit our GitHub project!}

  }

  }{

  %%%%%%%% LEFT COLUMN

  \title{\fontsize{50}{60}\selectfont GatorGrouper: \\ \mbox{Using Python} and GitHub for
  Team Formation and Assessment}
    %
  \vspace*{-.25in}
  \author{Gregory M.\@ Kapfhammer \\[.5in]
  \mbox{\faLink{} gregorykapfhammer.com} \\
  \faTwitter{} @GregKapfhammer \\
  \faGithubAlt{} github.com/gkapfham

  }

  \vspace*{-.5in}

  \section{Introduction}
  %
  Python programs are often complex and difficult to test.
  %
  But test coverage does not show that a method was adequately tested.
  %
  A method $m$ is psuedo-tested if a test passes even when $m$ is not run.

  \vspace*{.25in}

  Since it may be time-consuming and error-prone to manually detect
  psuedo-tested methods, Function-Fiasco automatically discovers them for
  engineers.

  \vspace*{-.25in}
  %
  \section{\faCog~Algorithms}
  %
  GatorGrouper forms teams with:

  \begin{itemize}

    \item Random
    \item Round-robin
    \item Kerninghan-Lin
    \item Genetic Algorithm

  \end{itemize}

  \vspace*{.5in}
  %
  Function-Fiasco performs these steps:

  \begin{enumerate}[leftmargin=.5in]

    \item Instrument all program methods
    \item Elide execution of chosen method
    \item Run the tests and observe behavior
    \item Run steps (2) and (3) for all methods
    \item Report the psuedo-tested methods

  \end{enumerate}

  \vspace*{.25in}
  %
  Steps are optimized, ensuring that the tool scales to large Python programs.
  %
  \vspace*{.5in}
  %

  %% This fills the space between the content and the logo
  \vfill

  \includegraphics[width=\textwidth]{img/CollegeLogo.eps}\\

  }{

  %%%%%%%% RIGHT COLUMN

  \vspace*{-.5in}
  %
  \section{\faCodeFork~Development}
  %
  We adopted these tools and processes:\\
  %
  \vspace*{-.5in}
  %
  \begin{itemize}[leftmargin=*]

    \item{Pytest for automated testing}
      %
    \item{Pyenv for language management}
      %
    \item{Pipenv for handling environments}
      %
    \item{Travis for integration and deployment}
      %
    \item{GitHub flow for project development}

  \end{itemize}

  \section{\faCloudUpload~Deployment}
  %
  You can use GatorGrouper these ways:\\
  %
  \vspace*{-.5in}
  %
  \begin{itemize}[leftmargin=*]

    \item{Command-line with local storage}
      %
    \item{Django-based web application}
      %
    \item{Leverage Amazon Elastic Beanstalk}

  \end{itemize}

  \section{\faDashboard~Future Work}
  %
  Add new features to GatorGrouper: \\
  %
  \vspace*{-.5in}

  \begin{itemize}[leftmargin=*]

    \item{Provide new ways to form teams}
      %
    \item{Accept and use feedback on teams}
      %
    \item{Leverage data from GatorGrader}
      %
    \item{Integrate with GitHub Classroom}

  \end{itemize}

  \vspace{.5em}
  %
  Considering both industry \& academia, use improved GatorGrouper to
  empirically study team effectiveness.

  \section{\faRocket~Action Steps}
  %
  Created with Python, GatorGrouper is a tool that supports team creation by
  balancing knowledge, skills, and conflicts.

  \vspace*{.25in}
  %
  If you would like support the development of GatorGrouper, please raise an
  issue on the tracker or create a pull request to add a new feature or bug fix.

  \vspace*{.25in}
  %
  See \faGithubAlt~\underline{GatorEducator/GatorGrader} for a related tool that
  automatically checks the work of writers and programmers.

  \section{\faThumbsOUp~Acknowledgements}
  %
  Check \faGithubAlt~\underline{GatorEducator/GatorGrouper} for a full list of
  project contributors.

  % \vspace*{.5in}
  % \includegraphics[width=\textwidth]{img/ComputerScience-Stack}

}

\end{document}
